\documentclass[a4paper,11pt]{article}
\usepackage[margin=3cm]{geometry}
\usepackage[utf8]{inputenc}
\usepackage{listings, verbatim}
\usepackage{graphicx}
\usepackage{amsmath, amsthm, amssymb, textcomp, alltt, tcolorbox, framed}
\renewcommand{\thesection}{}
\renewcommand{\thesubsection}{}
\newcommand{\code}[1]{{\fontfamily{pcr}\selectfont #1}}

\lstset{basicstyle={\fontfamily{pcr}\selectfont}}
\begin{document}

\title{Projet système digital}
\date{}
\maketitle

\section{I. Utilisation} 

Le simulateur fonctionne de la manière suivante~: une netlist est
compilée vers un fichier C, qui est ensuite compilé vers un fichier
exécutable. Celui-ci peut ensuite être éxécuté pour simuler la
netlist.


Pour compiler le compilateur, il suffit de faire \code{ocamlbuild
  compiler.native}~; celui-ci dépend de ocamllex et menhir comme le
simulateur du TP 1.


Pour l'utiliser, il suffit de faire \code{./compiler.native [options]
  fichier.net}~; le compilateur produira ensuite un fichier C
\code{fichier.c} et un fichier exécutable \code{fichier}. Les options
acceptées sont les suivantes~:

\begin{itemize}
\item{\code{-print}~: N'écrit que le code C du programme, sans faire
    appel à \code{gcc -O2} pour le compiler ensuite}

\item{\code{-iomode $n$}~: Choisit le mode d'entrée-sortie de la
    netlist compilée~; le mode 0 (par défaut) correspond à des
    entrées-sorties interactives, où l'exécutable demande la valeur de
    chaque variable à chaque cycle (en décimal pour les nappes de
    fils), et effectue également l'affichage en décimal à chaque fin
    de cycle. Le mode 1 correspond à des entrées-sorties consituées
    d'un caractère par bit d'entrée-sortie, qui doit être '0' ou
    '1'. Les sorties sont affichées dans le même format que l'entrée.}

\item{\code{-n $n$}~: Nombre de cycles à simuler ; si omis, supposé
    égal à $+\infty$.}

\end{itemize}

Note~: le compilateur peut prendre un temps assez long, principalement
dû à l'appel de \code{gcc}~; ainsi, sur une netlist contenant environ $10^4$
équations, le compilateur prend un temps de 0.2s pour produire un
fichier C, tandis que la compilation de ce fichier par \code{gcc}
prend de l'ordre de 1.5s, et 50s avec \code{gcc -O2}. \\

Le fichier exécutable prend lui autant d'arguments que le nombre de
\textsc{rom}s du circuit~; chacun étant un fichier binaire contenant le contenu
initial des \textsc{rom}s.


\section{II. Fonctionnement}
Le compilateur commence par ordonner les équations afin que celles-ci
s'exécutent dans le bon ordre (on n'utilise pas une variable avant de
la mettre à jour, avec un traitement spécial des registres~: ceux-ci
sont en effet modifiés à la fin). Ensuite, la traduction de chaque
équation est ajoutée dans le corps de la boucle principale du
programme.

Les nappes de fils sont stockées dans des entiers 64 bits~; en
particulier, cela implique que les nappes de fils de plus de 64 bits
ne sont pas supportées facilement. Par souci de simplicité, j'ai
choisi de ne pas supporter ces nappes, qui ne sont pas très utiles en
pratique.

\section{III. Optimisations}
On utilise bien sûr des opérations bit-à-bit lors d'opérations sur des
nappes de fils. De plus, on garde une table de permettant de savoir où
trouver chaque nappe de fils ou fil (i.e. dans quelle variable, et sa
position interne à cette variable), ce qui permet de drastiquement
diminuer le nombre d'opérations \textsc{slice}, \textsc{select} et \textsc{concat}. Le
compilateur optimise finalement l'appel d'opérateurs sur chaque fil
d'une nappe (ou de deux nappes) en un seul opérateur bit-à-bit, afin
de simplifier encore plus le code produit. Ainsi, un morceau de code
\textsc{Minijazz} de ce type~: \\
\begin{lstlisting}
or_n<n>(a:[n], b:[n]) = (o:[n]) where
  if n = 0 then
    o = []
  else
    o = (a[0] or b[0]).(or_n<n-1>(a[1..], b[1..]))
  end if
end where

main(a:[10], b:[10]) = (o:[word]) where
  o = or_n<10>(a, b)
end where
\end{lstlisting}
sera traduit en~: \\
\code{o = a | b;} \\
On effectue donc une simplification assez importante de la netlist de cette manière.


\end{document}